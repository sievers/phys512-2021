\documentclass[11pt]{article}
%\usepackage{graphics,graphicx}
%\usepackage{natbib}
%\usepackage{color}
%\citestyle{aas}
\usepackage{url}

\begin{document}

{\bf{Problem Set 6 for Matched Filters.  Due Friday November 5 at
    11:59 PM}}
\vskip 0.1in
Key will be getting LIGO data from:\\

{\small{
\url{https://www.gw-openscience.org/static/events/LOSC_Event_tutorial.zip}}}
%https://www.gw-openscience.org/static/events/LOSC\_Event\_tutorial.zip
\vskip 0.1in

While they include code to do much of this, please don't use it
(although you may look at it for inspiration) and instead write your
own.  You can look at/use simple\_read\_ligo.py that I have posted for
concise code to read the hdf5 files.  Feel free to have your code loop
over the events and print the answer to each part for that event.  In
order to make our life easy, in case we have to re-run your code
(which we should not have to do), please also have a variable at the
top of your code that sets the directory where you have unzipped the
data.  LIGO has two detectors (in Livingston, Louisiana, and Hanford,
Washington) and GW events need to be seen by both detectors to be
considered real.  Note that my read\_template functon returns the
templates for both Hanford and Livingston (th and tl).  

%When we mark, we want to be able to just point your script to my copy of
%the LIGO data without having to re-download it!

{\bf{Problem 1}}: 

Find gravitational waves!  Parts should include 

a) Come up with a noise model for the Livingston and Hanford detectors
separately.
Describe in comments how you go about doing this.  Please mention
something about how you smooth the power spectrum and how you deal
with lines (if at all).  Please also explain how you window the data
(you may want to use a window that has an extended flat period near
the center to avoid tapering the data/template where the signal is not small).

b) Use that noise model to search the four sets of events using a
matched filter.  The mapping between data and templates can be found
in the file BBH\_events\_v3.json, included in the zipfile.

c)  Estimate a noise for each event, and from the output of the
matched filter, give a signal-to-noise ratio for each event, both from
the individual detectors, and from the combined Livingston + Hanford
events.

d)  Compare the signal-to-noise you get from the scatter in the
matched filter to the analytic signal-to-noise you expect from your
noise model.  How close are they?  If they disagree, can you explain
why? 

e) From the template and noise model, find the frequency from each
event where half the weight comes from above that frequency and half
below.  

f) How well can you localize the time of arrival (the horizontal shift
of your matched filter).  The positions of gravitational wave events
are inferred by comparing their arrival times at different detectors.
What is the typical positional uncertainy you might expect given that
the detectors area a few thousand km apart? 

\vskip 0.05in
BONUS: You may find the Advent of Code enjoyable if you enjoy coding
challenges.  Starting December 1 each year, there is a new two-part problem
each day until the 25th.  Previous years are available on-line.  Part
2 on December 20th last year ({\url{https://adventofcode.com/2020/day/20}})
involved finding sea monsters in a puzzle, a perfect task for a
matched filter.  You can still do previous year's problems.  For the
bonus, find the sea monsters using a matched filter.  Submit your
code, and a screenshot from AoC showing two stars on the 20th.  

Note - to help you out, I've added my code that solves part 1
(dec\_20\_1.py).  If you choose to use it, you'll need to edit your
input to replace ``.''  with ``0 `` and ``\#'' with ``1 `` (note the
added spaces).  One (but not the only) way to do this is in emacs with
meta-x replace-string.  


\end{document}
